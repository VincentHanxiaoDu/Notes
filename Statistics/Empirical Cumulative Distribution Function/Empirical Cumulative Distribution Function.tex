\documentclass[12pt,a4paper]{article}
\title{Empirical Cumulative Distribution Function (eCDF)}
\author{Hanxiao Du}
\date{}

\usepackage{amsmath}
\usepackage{amssymb}
\usepackage{amsthm}
\usepackage{bbm}
\usepackage{pifont}
\usepackage{marvosym}
\usepackage{mathtools}
\usepackage{environ}
\usepackage{lipsum}
\usepackage{undertilde}
\usepackage{hyperref}
\usepackage{graphicx}
\hypersetup{colorlinks=true,
     linkcolor=black,urlcolor=blue}

\usepackage[total={6in, 10in}]{geometry}
\newtheorem{theorem}{Theorem}
\newtheorem{definition}{Definition}
\setlength{\parindent}{0pt}
\renewcommand{\qedsymbol}{$\blacksquare$}
\DeclareMathOperator*{\argmax}{\arg\!\max}
\DeclareMathOperator*{\argmin}{\arg\!\min}

\begin{document}

\maketitle

\section{Introduction}
In statistics, an empirical distribution function (commonly also called an empirical Cumulative Distribution Function, eCDF) is the distribution function associated with the empirical measure of a sample.

\begin{definition}[Empirical distribution function]
Let $(X_1, \cdots, X_n)$ be n independent, identically distributed real random variables with the common cumulative distribution function $F(t)$. Then, the empirical distribution function is defined as:
\begin{align}
\hat{F}_n(t) = \frac{\text{number of elements in the sample} \leq t}{n} = \frac{1}{n} \sum_{i=1}^n \mathbb{I}(X_i \leq t)
\end{align}
where $\mathbb{I}(X_i \leq t)$ is the indicator of event $X_i \leq t$.
\end{definition}
\begin{theorem}[$\hat{F}_n(t)$ is an unbiased estimator of $F(t)$]
\ \\
\begin{proof}
Let $(X_1, \cdots, X_n)$ be n independent, identically distributed real random variables with the common cumulative distribution function $F(t)$.
For a fixed $t$, since that $\mathbb{I}(X_i \leq t) = 1$ with probability $F(t)$ and $\mathbb{I}(X_i \leq t) = 0$ with probability $1-F(t)$ by the definition of $F(t)$, the indicator $\mathbb{I}(X_i \leq t)$ is a Bernoulli random variable with parameter $p=F(t)$. Therefore, $\forall i \in [n]$,
\begin{align}
\mathbb{E}[\mathbb{I}(X_i \leq t)] = \mathbb{P}[X_i \leq t] = F(t)
\end{align}

Furthermore,
\begin{align}
\mathbb{E}[\hat{F}_n(t)] &= \mathbb{E}[\frac{1}{n} \sum_{i=1}^n \mathbb{I}(X_i \leq t)]\\
&= \frac{1}{n} \sum_{i=1}^n \mathbb{E}[\mathbb{I}(X_i \leq t)]\\
&= \frac{1}{n} \sum_{i=1}^n F(t)\\
&= F(t)
\end{align}
Thus, $\hat{F}_n(t)$ is an unbiased estimator of $F(t)$.\\ 
\end{proof}
\end{theorem}

\end{document}
